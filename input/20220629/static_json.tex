This section presents the latest data format in JSON for 
each network electrical component for the competition.
%The data format is described in a top-down approach.
This section first starts by presenting the static Network object ---
the top level JSON object containing all the \textbf{static data}
(attributes marked with `S'/`B' in Section~\ref{sec:naming}) for each network component.
JSON objects associated with each individual component will be presented afterwards.

\subsection{Network}
\label{sec:network}
The Network object in JSON consists of: 
\begin{itemize}
    % \item A \texttt{"substation"} object\\
    % The object contains a list of substations, defined in~\ref{sec:substation}.
    \item A \texttt{"general"} object\\
    The object contains data specified for the general attributes in Section~\ref{nom:top-level}.
    \item A \texttt{"violation\_cost"} object\\
    The object contains specifications for various violation costs, defined in~\ref{sec:violation}.
    \item A \texttt{"bus"} array\\
    The object contains a JSON array of buses, defined in~\ref{sec:bus}.
    \item A \texttt{"shunt"} array\\
    The object contains a JSON array of shunts, defined in~\ref{sec:shunt}.
    \item A \texttt{"simple\_dispatchable\_device"} array\\
    The object contains a JSON array of simple producing \& consuming devices, defined in~\ref{sec:generator}.
    % \item A \texttt{"multi\_mode\_dispatchable\_device"} \hl{array}\\
    % The object contains a JSON array of multi-mode producing \& consuming devices, defined in~\ref{sec:multi-generator}.
    % \item A \texttt{"sub\_device"} \hl{array}\\
    % The object contains a JSON array of subdevices, defined in~\ref{sec:subdevice}.
    % \item A \texttt{"storage\_device"} \hl{array}\\
    % The object contains a JSON array of storage devices, defined in~\ref{sec:storage}.
    \item A \texttt{"ac\_line"} array\\
    The object contains a JSON array of AC lines, defined in~\ref{sec:acline}.
    \item A \texttt{"two\_winding\_transformer"} array\\
    The object contains a JSON array of two winding transformers, defined in~\ref{sec:2wtransformer}.
    \item A \texttt{"dc\_line"} array\\
    The object contains a JSON array of DC lines, defined in~\ref{sec:dcline}.
    \item A \texttt{"active\_zonal\_reserve"} array\\
    The object contains a JSON array of active regional/zonal reserve requirements, defined in~\ref{sec:reserve}.
    \item A \texttt{"reactive\_zonal\_reserve"} array\\
    The object contains a JSON array of reactive regional/zonal reserve requirements, defined in~\ref{sec:reserve}.
    \item A \texttt{"development"} object\\
    The object contains development data for ARPA-E Dataset Teams. This section is primarily reserved for development. Competitors are not required to process the JSON object. The object may or may not be presented. 
\end{itemize}

Below shows a high-level schematic description for the network object and the general object in JSON:
% \begin{verbatim}
% "network”: {
%   "general" : { "base_norm_mva" : 100.0 },
%   "violation_cost": { … },
%   "bus": [ … ],
%   "shunt": [ … ],
%   "simple_dispatchable_device": [ … ],
%   "multi_mode_dispatchable_device": [ … ],
%   "sub_device": [ … ],
%   "storage": [ … ],
%   "ac_line”: [ … ],
%   "two_winding_transformer”: [ … ],
%   "dc_line”: [ … ],   
%   "active_regional_reserve”: [ … ],
%   "reactive_regional_reserve”: [ … ],
%   "development" : { ... }
% }    
% \end{verbatim}
\begin{verbatim}
"network”: {
  "general" : { "base_norm_mva"   : 100.0,
                "timestamp_start" : "2012-01-10T05:00",
                "timestamp_stop"  : "2012-01-11T05:00" },
  "violation_cost": { … },
  "bus": [ … ],
  "shunt": [ … ],
  "simple_dispatchable_device": [ … ],
  "ac_line”: [ … ],
  "two_winding_transformer”: [ … ],
  "dc_line”: [ … ],   
  "active_zonal_reserve”: [ … ],
  "reactive_zonal_reserve”: [ … ],
  "development" : { ... }
}    
\end{verbatim}

\subsection{Violation Cost}
\label{sec:violation}
Below shows a schematic example on the static violation cost parameters for 
describing the costs/penalties for violating the bus active flows, bus reactive flows, 
and branch (thermal) MVA limits. 

\begin{verbatim}
{
 "p_bus_vio_cost" : 1000000.0,
 "q_bus_vio_cost" : 1000000.0,
 "s_vio_cost" : 500.0
}
\end{verbatim}


\subsection{Bus}
\label{sec:bus}
A bus represents a location point where equipment connected together.
It could be a physical bus bar of a substation, or 
a connection point between various equipment, 
or a virtual location/split-point resulting from
component aggregation or model transformation.
Attributes for buses are listed in Section~\ref{nom:bus}.
Below shows a schematic example on static parameters for buses.
\begin{verbatim}
{
  "uid" : "IDB0001",
  "vm_ub" : 1.1,
  "vm_lb" : 0.95,
  "active_reserve_uids" : ["IDRES0001", "IDRES0002"],
  "reactive_reserve_uids" : ["IDRES0010"],
  "area” : "MI",
  "zone” : "DET",
  "longitude" : -83.045753,
  "latitude" : 42.331429,
  "city" : "Detroit",
  "county" : "Wayne",
  "state" : "Michigan",
  "country" : "USA",
  "con_loss_factor" : 1.0,
  "base_nom_volt” : 100.0,
  "type” : "PQ",
  "initial_status" : {
     "vm" : 1.05,
     "va" : 0.0013
  }
},
...
\end{verbatim}


\subsection{Shunt}
\label{sec:shunt}
Bus shunts are bus components for modeling losses to ground, and 
usually will be defined as an admittance to ground.
% :
% $$
% 	I = -Y \cdot V,
% $$
% where $Y$ is the admittance, and $V$ is the voltage at the connecting point. Figure \ref{fig:shunt} illustrates a shunt.
% \begin{figure}[h]
%     \centering
% 	\includegraphics[width=0.25\linewidth]{img/shunt_new}
% 	\caption{\label{fig:shunt} A shunt.}
% \end{figure}
Attributes for shunts are listed in Section~\ref{nom:shunt}.
Below shows a schematic example on the static parameters for shunts.
\begin{verbatim}
{
    "uid" : "IDSH0001",
    "bus" : "IDB0002",
    "gs”  : 0.0,
    "bs”  : 0.001,
    "step_ub" : 2,
    "step_lb" : 0,
    "initial_status" : {
         "step" : 1
    }
},
…    
\end{verbatim}


\subsection{Dispatchable Devices: Simple Producing \& Consuming Devices}
\label{sec:generator}
Producing devices
produce active power and reactive power, and support voltage at target voltage set-point.
These devices can be classical synchronous generators, renewable generations, or supporting 
power from other lumped/aggregated areas. 
Consuming devices consume active power and reactive power.
These devices can be household demands, industrial loads, pump storage, or
supporting power to other lumped/aggregated areas. 
For the competition, a producing/consuming device is classified as a simple device if it
does not have switchable modes.
Attributes for simple producing \& consuming devices are listed in Section~\ref{nom:gen_single}.
Below shows a schematic example on the static parameters for simple producing \& consuming devices.

\begin{verbatim}
{
    "uid" : "IDG0001",
    "bus" : "IDB0002",
    "device_type" : "producer",
    "description" : "gas",
    "vm_setpoint” : 1.2,
    "startup_cost" : 24.0,
    "startup_states" : [ [-20.0, 3.0], [-10.0, 8.0], [0.0, 24.0] ], 
    "shutdown_cost" : 5.0,
    "startups_ub" : [ [0.0, 1.0, 1], [1.0, 3.0, 1] ],
    "energy_req_ub" : [ [0.0, 1.0, 2.0], [1.0, 3.0, 5.0] ],
    "energy_req_lb" : [ [0.0, 1.0, 0.0], [1.0, 3.0, 0.0] ],
    "on_cost" : 0.1,
    "in_service_time_lb" : 5.0, "down_time_lb" : 10.0,
    "p_ramp_up_ub" : 10.0, "p_ramp_down_ub" : 10.0, 
    "p_startup_ramp_ub" : 5.0, "p_shutdown_ramp_ub" : 5.0,
    "initial_status": {
        "on_status": 0,
        "p" : 0.0,
        "q" : 0.0,
        "accu_down_time" : 10.0,
        "accu_up_time" : 0.0
    },
    "q_linear_cap" : 0, 
    "q_bound_cap" : 1, 
    "p_reg_res_up_ub" : 1.0, "p_reg_res_down_ub" : 1.0,
    "p_syn_res_ub" : 2.0, "p_nsyn_res_ub" : 4.0, 
    "p_ramp_res_up_online_ub" : 2.0, "p_ramp_res_down_online_ub" : 2.0,
    "p_ramp_res_up_offline_ub" : 1.0, "p_ramp_res_down_offline_ub" : 0.0,
    "q_0_ub" : 5.0,
    "q_0_lb" : 0.0,
    "beta_ub" : 2.0,
    "beta_lb" : -0.5    
},
…     
\end{verbatim}

% \subsection{Dispatchable Devices: Multi-mode Producing \& Consuming Devices}
% \label{sec:multi-generator}
% A producing/consuming device is classified as a multi-mode device if it
% has more than one switchable modes.
% Attributes for multi-mode producing \& consuming devices are listed in Section~\ref{nom:gen_multi}.
% Below shows a schematic example on the static parameters for multi-mode producing \& consuming devices.


% \begin{verbatim}
% {
%     "uid" : "IDG0015",
%     "bus" : "IDB0004",
%     "device_type" : "producer",
%     "vm_setpoint” : 1.0,
%     "startup_cost" : 4.0, "shutdown_cost" : 1.0,
%     "startups_ub" : [ [0.0, 1.0, 1], [1.0, 4.0, 2] ],
%     "energy_req_ub" : [ [0.0, 1.0, 10.0], [1.0, 4.0, 15.0] ],
%     "energy_req_lb" : [ [0.0, 1.0, 0.0],  [1.0, 4.0, 1.0] ],
%     "in_service_time_lb" : 2.0, "down_time_lb" : 0.0,
%     "p_startup_ramp_ub" : 4.0, "p_shutdown_ramp_ub" : 4.0,
%     "initial_status": {
%         "select_mode" : "IDG0015-M1",
%         "p" : 1.0,
%         "q" : 0.7,
%         "accu_up_time" : 10.0,
%         "accu_down_time" : 0.0
%     },
%     "q_linear_cap" : 0, 
%     "q_bound_cap" : 1, 
%     "mode_num" : 2,
%     "subdevice_uids" : ["IDG0015-SD1"],
%     "modes" : [
%         {
%         "uid" : "IDG0015-M1",
%         "mode_transition_uids" : ["IDG0015-M1", "IDG0015-M2"],
%         "on_subdevice_uids" : [],
%         "on_cost" : 1.0,
%         "mode_transit_cost" : {
%           "IDG0015-M1" : 0.0,
%           "IDG0015-M2" : 1.0,
%         },
%         "p_ramp_up_ub" :  8.0,
%         "p_ramp_down_ub" : 8.0,
%         "dwell_time_ub" : 5.0, "dwell_time_lb" : 2.0,        
%         "initial_status" : {
%             "accu_up_time" : 10.0
%         },
%         "p_reg_res_up_ub" : 1.0, "p_reg_res_down_ub" : 1.0,
%         "p_syn_res_ub" : 3.0, "p_nsyn_res_ub" : 4.0, 
%         "p_ramp_res_up_online_ub" : 8.0, "p_ramp_res_down_online_ub" : 8.0,
%         "p_ramp_res_up_offline_ub" : 0.0, "p_ramp_res_down_offline_ub" : 0.0,
%         "q_0_ub" : 5.0,
%         "q_0_lb" : 0.0,
%         "pg_ratio_ub" : 2.0,
%         "pg_ratio_lb" : 0.5
%         },
%         {
%         "uid" : "IDG0015-M2",
%         "mode_transition_uids" : ["IDG0015-M1", "IDG0015-M2"],
%         "on_subdevice_uids" : ["IDG0015-SD1"],
%         "on_cost" : 0.0,
%         "mode_transit_cost" : {
%           "IDG0015-M1" : 1.0,
%           "IDG0015-M2" : 0.0,
%         },
%         "p_ramp_up_ub" :  10.0,
%         "p_ramp_down_ub" : 10.0,
%         "dwell_time_ub" : 240.0, "dwell_time_lb" : 10.0,
%         "initial_status" : {
%             "accu_up_time" : 0.0
%         },
%         "p_reg_res_up_ub" : 2.0, "p_reg_res_down_ub" : 2.0,
%         "p_syn_res_ub" : 7.0, "p_nsyn_res_ub" : 0.0, 
%         "p_ramp_res_up_online_ub" : 10.0, "p_ramp_res_down_online_ub" : 10.0,
%         "p_ramp_res_up_offline_ub" : 10.0, "p_ramp_res_down_offline_ub" : 0.0,
%         "q_0_ub" : 5.0,
%         "q_0_lb" : 0.0,
%         "pg_ratio_ub" : 2.0,
%         "pg_ratio_lb" : 0.5    
%         }
%     ]
% }
% …     
% \end{verbatim}

% \subsection{Producing Devices: Subdevice Units}
% \label{sec:subdevice}
% A subdevice unit can be coarsely viewed as 
% a device without direct power/energy connection
% to the grid and belongs to a multi-mode producing \& consuming device.
% Attributes for subdevice units are listed in Section~\ref{nom:gen_sub}.
% Below shows a schematic example on the static parameters for subdevice units.


% \begin{verbatim}
% {
%     "uid" : "IDG0015-SD1",
%     "startup_cost" : 10.0, "shutdown_cost" : 0.0,
%     "startups_ub" : [ [0.0, 4.0, 5] ],
%     "on_cost" : 1.0,
%     "in_service_time_lb" : 0.0, "down_time_lb" : 5.0,
%     "initial_status": {
%         "on_status": 0,
%         "accu_down_time" : 20.0,
%         "accu_up_time" : 0.0
%     },    
% }
% …     
% \end{verbatim}

% \subsection{Storage Device Units}
% \label{sec:storage}
% \begin{todo}[]{}
% TBD 
% \end{todo}


\subsection{AC Line}
\label{sec:acline}
An AC line connects network devices in two different points of the network. 
It has has two sides: \emph{from} (fr) side and the \emph{to} side.
In general, power flows are bi-directional. 
Readers should always refer to the formulation document for the +ve and -ve signs
requirements for indicating directions in their solutions,
and consulate the competition team if questions arise.
% For ease of illustration, we assume the current is positive when flowing from side from to side to.
% The currents $I_1$ (from side) and $I_2$ (to side) are given by:
% \begin{align}
% 	I_1 &= Y_1 \cdot V_1 + \frac{1}{Z} (V_1 - V_2) \\
% 	I_2 &= - Y_2 \cdot V_2 + \frac{1}{Z} (V_1 - V_2),
% \end{align}
% where $Y_1 = G_1 + jB_1$ and $Y_2 = G_2 + jB_2$ are the admittances at the
% from side and to side, respectively, $Z = R + jX$ is the line impedance, and $V_1$ and $V_2$ are the 
% voltages at the connecting points. 
% Line current limits are monitored at both ends. 
% %
% Figure \ref{fig:line} illustrates an AC line.
% %between nodes 1 at side from and node 2 at side to (represented by the black points at the ends).
% The power $S_i$ (where $i \in \{1,2\})$ is given by:
% \begin{equation}
% 	S_i = \bar{I}_i \cdot V_i
% \end{equation}
% Note that software implementations can choose any current/power flow directions, 
% and differences in signs are expected. 
% %Users will need to flip the sign of the current variables or power variables if the direction %is changed from "left to right" to "right to left".
% \begin{figure}[h]
%     \centering
% 	\includegraphics[width=0.75\linewidth]{img/line_new}
% 	\caption{\label{fig:line} Illustration of an AC line.}
% \end{figure}
Attributes for lines are listed in Section~\ref{nom:line}.
Below shows a schematic example on the static parameters for lines.

Static component data example:
% "standby_cost" : 1.0,
\begin{verbatim}
{
    "uid" : "IDAC0001",
    "fr_bus” : "IDB0021",
    "to_bus” : "IDB0025",
    "r” : 0.001, "x” : 0.05, 
    "b" : 0.0001,
    "mva_ub_nom” : 10.0, "mva_ub_sht” : 15.0, "mva_ub_em” : 100.0,    
    "connection_cost" : 0.01,
    "disconnection_cost" : 0.01,
    "initial_status" : {
        "on_status" : 1
    }
    "additional_shunt" : 1,
    "g_fr” : 0.0, "b_fr” : 0.01,
    "g_to” : 0.0, "b_to” : 0.0    
}
…     
\end{verbatim}


\subsection{Two Winding Transformer}
\label{sec:2wtransformer}
A two winding transformer connects devices located at two different voltage levels of the network. 
% We denote the voltage level at the \emph{from} side to be $\text{VL}_1$, and 
% the voltage level at the \emph{to} side to be $\text{VL}_2$.
% Their associated nominal voltage values are denoted to be 
% $V_1^\text{nom}$ and $V_2^\text{nom}$ respectively.
Transformers are usually equipped with taps on their winding to adjust the voltage transformation or the phase angles through the transformer.
% For a specific tap $k$, we denote the voltage magnitude ratio for a transformer to be $r_k$, phase shift to be $\delta_k$, 
% impedance to be $Z_k = R_k + j X_k$, and admittance to be $Y_k = G_k + j B_k$.
% Figure \ref{fig:2windingtransformer} illustrates a two winding transformer.
% %----------------------------------------------------------------------------------------
% \begin{figure}[h]
%     \centering
% 	\includegraphics[width=0.75\linewidth]{img/2wtranf_T}
% 	\caption{\label{fig:2windingtransformer} Illustration of a two winding transformer.}
% \end{figure}
% %----------------------------------------------------------------------------------------
% %We define \emph{side 1} as the high voltage side, and \emph{side 2} as the low voltage side. 
% Similar to AC transmission line, without loss of generality, 
% we assume the current is positive when flowing from the from side to the to side (i.e. left to right). 
% The currents $I_1$ (from side) and $I_2$ (to side) are given by:
% \begin{align}
% 	I_1 &= \bar{\rho_k} \cdot \big( Y_k \cdot V_1' + \frac{1}{Z_k} (V_1' - V_2) \big) \\
% 	I_2 &= \frac{1}{Z_k} ( V_1' - V_2), \mbox{ where} \\
%  \rho_k &=   \big( (\frac{V_2^\text{nom}}{V_1^\text{nom}}) \cdot r_k \big) \cdot e^{j \delta_k} \mbox{ and } V_1' = V_1 \rho_k 
% \end{align}
% where $Y_k = G_k + j B_k$ is the admittance of the transformer on tap $k$,
% $Z_k = R_k + j X_k$ is the impedance of the transformer on tap $k$,
% %\begin{equation}
% %	Z_k' = \big( R \cdot (1 + \frac{R_k}{100})\big) + j\big( X \cdot (1 + \frac{X_k}{100}) \big),
% %\end{equation}
% $V_1$ and $V_2$ are the voltages at the connecting buses in VL$_1$ and VL$_2$, respectively. 
% We use $V_1^\text{nom}$ and $V_2^\text{nom}$ to denote the 
% nominal voltage magnitudes at sides 1 and 2 of the transformer, respectively. 
% %Figure \ref{fig:2windingtransformer} illustrates a two winding transformer, where the points labeled 1 and 2 represent sides 1 and 2 connected by the branch.
% The power $S_i$ at sides $i$ ($i \in \{1,2\})$ is given by:
% \begin{equation}
% 	S_i = \bar{I}_i \cdot V_i
% \end{equation}
% Similar to AC lines, transformers also have current limits.
Similar to AC lines, power flows are also bi-directional for transformers. 
Readers should always refer to the formulation document for the +ve and -ve signs
requirements for indicating directions in their solutions,
and consulate the competition team if questions arise.
Attributes for two winding transformers are listed in Section~\ref{nom:transformer}.
Below shows a schematic example on the static parameters for transformers.

Static component data example:
%     "standby_cost" : 2.0,
\begin{verbatim}
{
    "uid" : "ID2T0001",
    "fr_bus” : "IDB0021", "to_bus” : "IDB0025",
    "r” : 0.005, "x” : 0.01, 
    "b” : 0.0001,
    "tm_lb” : 0.98, "tm_ub” : 1.02, 
    "ta_lb” : 0.0, "ta_ub” : 0.0, 
    "mva_ub_nom” : 100.0, "mva_ub_sht” : 250.0, "mva_ub_em” : 500.0,
    "connection_cost" : 0.1,
    "disconnection_cost" : 0.1,
    "initial_status" : {
          "tm" : 1.0, "ta" : 0.0,
          "on_status" : 1
    }
    "additional_shunt" : 1,
    "g_fr” : 0.0, "b_fr” : 0.0,
    "g_to” : 0.0, "b_to” : 0.01    
},
…     
\end{verbatim}


\subsection{DC Line}
\label{sec:dcline}
A DC line connects network devices at two different points of the network, with an AC-DC converter on each side.
% If the converter converts power from AC to DC, it is a rectifier. 
% If the converter converts power from DC to AC, it is an inverter.
Power flowing through the DC line are first
converted from AC to DC by a rectifier, before delivering power 
back to the network in AC by an inverter.
For simplicity, the competition models DC lines as two generators 
on each side with active power coupling.
Similar to an AC line, DC line has two sides ---  \emph{from} (fr) side and \emph{to} side.
% %with \emph{side 1} as the from side, and \emph{side 2} as the to side.
% For consistency, we assume the current flows from side 1 (from side) to side 2 (to side).
% %Therefore, we also assume side 1  will connect to a rectifier, and side 2 will be connecting to an inverter. 
% The DC voltages $V_1$ (at from side) and $V_2$ (at to side) of a DC line are linked by:
% \begin{align}
% 	V_1 &= V_2 + R I 
% \end{align}
% where $R$ is the DC line resistance, and $I$ is the DC current flowing through the DC line. 
% The DC \hyperref[sec:power]{power} $S_i$ at side $i$ ($i \in \{1,2\})$ is given by:
% \begin{equation}
% 	S_i = I \cdot V_i
% \end{equation}
% %Again, many softwares/implementations may assume a different current/power flow directions. 
% %Users will need to flip the sign of the current/power variables if the direction is changed from "left to right" to "right to left".
% %Active power loss for the converters can also be modeled by: 
% %\begin{equation}
% % \min(\max(l_{\mbox{min}}, \lvert c_1 P + c_0 \rvert), l_{\mbox{max}} )
% %l_a I^2 + l_b I + l_c 
% %\end{equation}
% %where $I$ is the current flowing through the converter, $l_{a}, l_b$, and $l_c$ are the loss
% %function parameters.
% {\color{red} Extra modeling data for specific converter control mechanism (e.g. VSC or LCC controls) can also be added to the data format if needed. }
% Figure \ref{fig:dc_line} illustrates a DC line connecting to two converters.
% \begin{figure}[h]
% \centering
% 	\includegraphics[width=0.75\linewidth]{img/dc_line}
% 	\caption{\label{fig:dc_line} Illustration of a DC line.}
% \end{figure}
% DC line current limits can also be monitored on DC lines at both ends. 
Attributes for DC lines are listed in Section~\ref{nom:dcline}.
Below shows a schematic example on the static parameters for DC lines.
%    "standby_cost" : 1.0,
%       "connection_cost" : 0.01,
%       "disconnection_cost" : 0.01,
%      "on_status" : 1,
% "pdc_fr_lb” : -1.0,
\begin{verbatim}
{
   "uid" : "IDDC0001",
   "fr_bus” : "IDB0009", 
   "to_bus” : "IDB0010",
   "pdc_ub” : 1.0, 
   "qdc_fr_lb” : -1.0, "qdc_fr_ub” : 1.0,
   "qdc_to_lb” : -0.5, "qdc_to_ub” : 0.5,
   "initial_status" : {
      "pdc" : 1.0, 
      "qdc_fr" : 1.0, "qdc_to" : 0.5 
   }
},
…    
\end{verbatim}



\subsection{Zonal Reserve}
\label{sec:reserve}
A reserve zone is a partition of the grid containing a collection of resources with individual reserve requirements.
In practice, each reserve zone is designed to handle contingencies 
and daily regulation activities within the area.
In the competition, we consider two types of reserve zones --- active reserve zones and reactive reserve zones. 
% \begin{itemize}
%     \item A \texttt{"uid"} object\\
%     The value of the object must be a unique ID (string) identifying the reserve region.    
%     \item A \texttt{"type"} object\\
%     The value of the object is a string indicating the type of the reserve: SPIN for spinning reserve, RAMP-UP for ramp up reserve, RAMP-DOWN for ramp down reserve, REG-UP for regulation up reserve, REG-DOWN for regulation down reserve, CONT for contingency reserve, REACT-UP for reactive up reserve, and REACT-DOWN for reactive down reserve.   
%     \item A \texttt{"reserve\_required"} object\\
%     The value of the object specifies the amount of active reserves required.
%     \item A \texttt{"res\_vio\_block"} object\\
%     The violation cost blocks for describing the penalties when violating reserve products.
% \end{itemize}
Below shows a schematic example on the static parameters for the active reserve zones. 
%Time varying parameters will be shown in the next section.

\begin{verbatim}
{
  "uid" : "IDRES0001",
  "REG_UP" : 0.03,
  "REG_DOWN" : 0.03,
  "SYN"  :  0.30,
  "NSYN" :  0.70,
  "REG_UP_vio_cost" : 1244.0,
  "REG_DOWN_vio_cost" : 1244.0,
  "SYN_vio_cost" : 305.0,
  "NSYN_vio_cost" : 24.0,  
  "RAMPING_RESERVE_UP_vio_cost" : 0.10,  
  "RAMPING_RESERVE_DOWN_vio_cost" : 0.10
},
...
\end{verbatim}

Below shows a schematic example on the static parameters for the reactive reserve zones. 
\begin{verbatim}
{
  "uid" : "IDRES0010",
  "REACT_UP_vio_cost" : 24.0,
  "REACT_DOWN_vio_cost" : 24.0
},
...
\end{verbatim}

% \begin{verbatim}
% {
%  "p_vio_cost" : 1000000.0,
%  "q_vio_cost" : 1000000.0,
%  "p_bus_vio_cost" : 1000000.0,
%  "q_bus_vio_cost" : 1000000.0,
%  "v_bus_vio_cost" : 1000.0,
%  "mva_branch_vio_cost" : 500.0,
% }
% \end{verbatim}



