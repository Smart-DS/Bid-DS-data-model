\label{sec:solution_format}
This section presents the latest data format in JSON for the 
communicating the solution data back to the ARPA-E competition team.
The section first presents the top level JSON object --- the
object for storing all the solution/output data 
%(attributes marked with `O'/`B' in Section~\ref{sec:naming})
for each network component.
JSON objects associated with each individual component will be presented afterwards.


\subsection{Solution Output Top Level}
\label{sec:output_data}
% \begin{verbatim}
% "solution_output”: {
%   "bus": [ … ],
%   "shunt": [ … ],
%   "simple_dispatchable_device": [ … ],
%   "multi_mode_dispatchable_device": [ … ],
%   "sub_device": [ … ],
%   "storage": [ … ],
%   "ac_line”: [ … ],
%   "two_winding_transformer”: [ … ],
%   "dc_line”: [ … ],   
%   "active_regional_reserve”: [ … ],
%   "reactive_regional_reserve”: [ … ]
% }    
% \end{verbatim}
\begin{verbatim}
"time_series_solution”: {
   "bus": [ … ],
   "shunt": [ … ],
   "simple_dispatchable_device": [ … ],
   "ac_line”: [ … ],
   "two_winding_transformer”: [ … ],
   "dc_line”: [ … ]
}    
\end{verbatim}


\subsection{Bus}
Attributes for buses are listed in Section~\ref{nom:bus}.
Below shows a schematic example on output parameters for buses.
\begin{verbatim}
{
  "uid" : "IDB0001",
  "vm" : [1.0, 0.99, 0.98],
  "va" : [0.0, 0.0125, 0.0233]
},
...
\end{verbatim}

\subsection{Shunt}
Attributes for shunts are listed in Section~\ref{nom:shunt}.
Below shows a schematic example on the output parameters for shunts.
\begin{verbatim}
{
    "uid" : "IDSH0001",
    "on_status": [1, 1, 0],
    "step" : [1, 1, 0]
},
…    
\end{verbatim}

\subsection{Dispatchable Devices: Simple Producing \& Consuming Devices}
Attributes for simple producing \& consuming devices are listed in Section~\ref{nom:gen_single}.
Below shows a schematic example on the output parameters for the simple devices.

\begin{verbatim}
{
    "uid" : "IDG0001",
    "on_status" : [1, 1, 1],
    "p" : [0.0, 2.0, 7.0],
    "q" : [0.0, 1.0, 4.0],
    "p_reg_res_up" : [0.0, 0.5, 0.5],
    "p_reg_res_down" : [0.0, 0.5, 1.0],
    "p_syn_res" : [0.0, 0.5, 1.0],
    "p_nsyn_res" : [0.0, 0.5, 1.0],
    "p_ramp_res_up_online" : [0.0, 1.0, 0.5],
    "p_ramp_res_down_online" : [0.0, 1.0, 2.0],
    "p_ramp_res_up_offline" : [1.0, 0.0, 0.0],
    "p_ramp_res_down_offline" : [0.0, 0.0, 0.0],
    "q_res_up" : [0.0, 1.0, 2.0],
    "q_res_down" : [0.0, 1.0, 2.0]
},
…     
\end{verbatim}

\subsection{AC Transmission Line}
Attributes for AC transmission lines are listed in Section~\ref{nom:line}.
Below shows a schematic example on the output parameters for lines.

\begin{verbatim}
{
    "uid" : "IDAC0001",
    "on_status" : [0, 1, 1]
},
…     
\end{verbatim}



\subsection{Two Winding Transformer}
Attributes for two winding transformers are listed in Section~\ref{nom:transformer}.
Below shows a schematic example on the output parameters for transformers.

\begin{verbatim}
{
    "uid" : "ID2T0001",
    "tm"    : [1.0, 1.05, 1.0],
    "ta"    : [0.0, 0.0, 0.0],
    "on_status" : [1, 1, 0]
},
…     
\end{verbatim}

\subsection{DC Line}
Attributes for DC lines are listed in Section~\ref{nom:dcline}.
Below shows a schematic example on the output parameters for DC lines.
\begin{verbatim}
{
   "uid" : "IDDC0001",
   "on_status" : [1, 1, 1],   
   "pdc_fr” : [0.5, 0.6, 0.7],
   "qdc_fr” : [0.0, 0.0, 0.1],
   "qdc_to” : [0.0, 0.0, -0.1]
},
…    
\end{verbatim}



% \subsection{Regional Reserve}
% Attributes for regional reserves are listed in Section~\ref{nom:reserves}.
% Below shows a schematic example on the output parameters for active reserve zones.
% \begin{verbatim}
% {
%   "uid" : "IDRES0001",
%   "REG_UP" : [0.10, 0.10, 0.10],
%   "REG_DOWN" : [0.05, 0.03, 0.05],
%   "SYN" : [0.40, 0.40, 0.60],
%   "NSYN" : [1.00, 1.00, 1.00],  
%   "RAMPING_RESERVE_UP" : [2.00, 3.00, 4.00],  
%   "RAMPING_RESERVE_DOWN" : [2.00, 3.00, 1.00]
% },
% ...
% \end{verbatim}

% Below shows a schematic example on the output parameters for reactive reserve zones.
% \begin{verbatim}
% {
%   "uid" : "IDRES0010",
%   "REACT_UP" : [0.02, 0.02, 0.01],
%   "REACT_DOWN" : [0.02, 0.02, 0.01]
% },
% ...
% \end{verbatim}

