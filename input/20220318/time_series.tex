\label{sec:time_series}
This section presents the latest data format in JSON for the 
time series input data for the competition.
The section first presents the global time-series JSON object --- the
top level JSON object for storing all the time series data 
(attributes marked with `T'/`B' in Section~\ref{sec:naming})
for each network component.
JSON objects associated with each individual component will be presented afterwards.


\subsection{Time series input}
\label{sec:input_data}
The time series JSON object consists of:
\begin{itemize}
    \item A \texttt{general} object\\
    The object contains general data listed
    in Section~\ref{nom:top-level}.
    For time series, these include the number of time periods (\texttt{time\_periods}) and the duration
    between each of the time point (\texttt{interval\_duration}).
    \item A \texttt{"simple\_dispatchable\_device"} array\\
    The object contains a JSON array of simple producing \& consuming devices, defined in~\ref{sec:generator_time}.
    \item A \texttt{"multi\_mode\_dispatchable\_device"} array \\
    The object contains a JSON array of multi-mode producing \& consuming devices, defined in~\ref{sec:multi-generator_time}.
    \item A \texttt{"subdevice"} array\\
    The object contains a JSON array of subdevices, defined in~\ref{sec:subdevice_time}.
    \item A \texttt{"storage"} array\\
    The object contains a JSON array of storage devices, defined in~\ref{sec:storage_time}.
    \item A \texttt{"active\_regional\_reserve"} array\\
    The object contains a list of active regional reserve requirements, defined in~\ref{sec:reserve_time}.
    \item A \texttt{"reactive\_regional\_reserve"} array\\
    The object contains a list of reactive regional reserve requirements, defined in~\ref{sec:reserve_time}.    
    \item A \texttt{"development"} object\\
    The object contains development data for ARPA-E Dataset Teams. This section is primarily reserved for development and competitors are not required to process the JSON object. The object may or may not be presented. 
    % \item A \texttt{regional\_reserve} object\\
    % The object contains a list of regional reserve objects, defined in~\ref{sec:reserve_time}
\end{itemize}

\begin{verbatim}
"time_series_input”: {
   "general" : {"time_periods": 3, "interval_duration": [1.0, 2.0, 1.0]},   
   "simple_dispatchable_device" : [ ... ],
   "multi_mode_dispatchable_device" : [ ... ],
   "subdevice" : [ ... ],
   "storage" : [ ... ],
   "active_regional_reserve" : [ ... ],
   "reactive_regional_reserve" : [ ... ],
   "development" : { ... }
}    
\end{verbatim}

% \begin{verbatim}
% "time_series_input”: {
%   "time_data" : {"time_periods": 3, "interval_duration": 1.0},    
%   "dispatchable_device": [ … ],
%   "regional_reserve": [ … ],
%   "weather": [
%      {
%      "region_uid" : "GA_AT",
%      "wind_sp" : [ ... ],
%      "wind_dir": [ ... ],
%      "solar_rad" : [ ... ], 
%      "cloud_cover" : [ ... ],
%      ...   
%      }
%   ]
% }    
% \end{verbatim}

\subsection{Dispatchable Devices: Simple Producing \& Consuming Devices}
\label{sec:generator_time}
Below shows a schematic example on the time varying parameters for simple producing \& consuming devices.
\begin{verbatim}
{
    "uid": "IDG0001",
    "on_status_ub" : [1, 1, 1],
    "on_status_lb" : [0, 0, 1],
    "p_lb" : [0.0, 0.0, 0.0], "p_ub" : [0.0, 5.0, 10.0],
    "q_lb" : [0.0, 0.0, 0.0], "q_ub" : [0.0, 5.0, 10.0],
    "cost" : [ 
               [[0.0, 0.0]],
               [[0.1, 5.0]],
               [[0.1, 5.0], [0.3, 5.0]] 
             ]
}, 
... 
\end{verbatim}

\subsection{Dispatchable Devices: Multi-mode Producing \& Consuming Devices}
\label{sec:multi-generator_time}
Below shows a schematic example on the time-varying parameters for multi-mode producing \& consuming devices.
\begin{verbatim}
{
    "uid": "IDG0015",
    "on_status_ub" : [1, 1, 1],
    "on_status_lb" : [0, 0, 0],
    "modes" : [
        {
        "uid" : "IDG0015-M1",
        "select_ub" : [1, 1, 1],
        "select_lb" : [0, 0, 0],
        "p_lb" : [0.0, 0.0, 0.0], "p_ub" : [20.0, 20.0, 20.0],
        "q_lb" : [0.0, 0.0, 0.0], "q_ub" : [20.0, 20.0, 20.0],
        "cost" : [ 
               [[1.0, 20.0]],
               [[1.0, 20.0]],
               [[1.0, 20.0]] 
             ]    
        },
        {
        "uid" : "IDG0015-M1",
        "select_ub" : [1, 1, 1],
        "select_lb" : [0, 0, 0],
        "p_lb" : [0.0, 0.0, 0.0], "p_ub" : [5.0, 10.0, 8.0],
        "q_lb" : [0.0, 0.0, 0.0], "q_ub" : [5.0, 10.0, 8.0],
        "cost" : [ 
               [[0.1, 5.0]],
               [[0.1, 5.0], [0.2, 5.0]],
               [[0.1, 5.0], [0.2, 3.0]] 
             ]    
        }    
    ] 
}, 
... 
\end{verbatim}

\subsection{Producing Devices: Subdevice Units}
\label{sec:subdevice_time}
Below shows a schematic example on the time-varying parameters for subdevice units.
\begin{verbatim}
{
    "uid": "IDG0015-SD1",
    "on_status_ub" : [1, 1, 1],
    "on_status_lb" : [0, 0, 0]
}, 
... 
\end{verbatim}

\subsection{Storage Devices}
\label{sec:storage_time}
Below shows a schematic example on the time-varying parameters for storage units.
\begin{todo}[]{}
TBD 
\end{todo}

\subsection{Regional Reserve}
\label{sec:reserve_time}
% \begin{itemize}
%     \item A \texttt{"uid"} object\\
%     The value of the object must be a unique ID (string) identifying the reserve region.    
%     \item A \texttt{"type"} object\\
%     The value of the object is a string indicating the type of the reserve: SPIN for spinning reserve, RAMP-UP for ramp up reserve, RAMP-DOWN for ramp down reserve, REG-UP for regulation up reserve, REG-DOWN for regulation down reserve, CONT for contingency reserve, REACT-UP for reactive up reserve, and REACT-DOWN for reactive down reserve.   
%     \item A \texttt{"reserve\_required"} object\\
%     The value of the object specifies the amount of active reserves required.
%     \item A \texttt{"res\_vio\_block"} object\\
%     The violation cost blocks for describing the penalties when violating reserve products.
% \end{itemize}
Below shows a schematic example on the time-varying parameters for the active reserve zones. 
%Time varying parameters will be shown in the next section.

\begin{verbatim}
{
  "uid" : "IDRES0001",
  "REG_UP" : [0.03, 0.03, 0.04],
  "REG_DOWN" : [0.03, 0.03, 0.04],
  "SYN" : [0.30, 0.30, 0.50],
  "NSYN" : [0.70, 0.70, 1.0],  
  "RAMPING_RESERVE_UP" : [1.00, 1.00, 2.00],  
  "RAMPING_RESERVE_DOWN" : [1.00, 1.00, 0.50]
},
...
\end{verbatim}

% \begin{verbatim}
% {
%   "uid" : "IDRES0001",
%   "REG_UP" : 0.03,
%   "REG_DOWN" : 0.03,
%   "SPIN" : 0.30,
%   "NON_SPIN" : 0.70,  
%   "FLEXI_RAMP_UP" : ,  
%   "FLEXI_RAMP_DOWN" : , 
%   "REG_UP_cost" : ,
%   "REG_DOWN_cost" : ,
%   "SPIN_cost" : ,
%   "NON_SPIN_cost" : ,  
%   "FLEXI_RAMP_UP_online_cost" : ,  
%   "FLEXI_RAMP_DOWN_online_cost" : ,
%   "FLEXI_RAMP_UP_offline_cost" : ,
%   "FLEXI_RAMP_DOWN_offline_cost" : 
% },
% ...
% \end{verbatim}

Below shows a schematic example on the time-varying parameters for the reactive reserve zones. 
\begin{verbatim}
{
  "uid" : "IDRES0010",
  "REACT_UP" : [0.01, 0.01, 0.005],
  "REACT_DOWN" : [0.01, 0.01, 0.005]
},
...
\end{verbatim}

% \begin{verbatim}
% {
% "uid": "IDG0001",
% "on_status_ub" : [1, 1, 1],
% "on_status_lb" : [0, 0, 1],
% "config" : [
%  {"uid" : "IDG0001-C1",
%  "select_ub" : [1, 1, 1],
%  "select_lb" : [0, 0, 0],
%  "pg_lb" : [0.0, 0.0, 0.0], "pg_ub" : [0.0, 1.0, 5.0],
%  "qg_lb" : [0.0, 0.0, 0.0], "qg_ub" : [0.0, 0.5, 2.5],
%  "cost" : [ [[0.0, 0.0]],
%             [[0.1, 1.0]],
%             [[0.05, 4.0], [0.1, 1.0]] ],
%  "energy_ub" : [0.0, 0.0, 0.0],
%  "energy_lb" : [0.0, 0.0, 0.0],
%  "pg_ext"    : [0.0, 0.0, 0.0]
%  },
%  {"uid" : "IDG0001-C2",
%  "select_ub" : [0, 1, 1],
%  "select_lb" : [0, 0, 0],
%  "pg_lb" : [0.0, -2.0, -7.0], "pg_ub" : [0.0, 2.0, 7.0],
%  "qg_lb" : [0.0, -1.0, -4.0], "qg_ub" : [0.0, 1.0, 4.0],
%  "cost" : [ [[0.0, 0.0]],
%             [[0.05, 2.0], [0.1, 2.0]],
%             [[0.05, 7.0], [0.1, 7.0]] ],
%  "energy_ub" : [10.0, 12.0, 13.0],
%  "energy_lb" : [0.0, 0.0, 0.0],
%  "pg_ext"    : [0.0, 0.0, 0.5]
%  }
% ]
% }, 
% ... 
% \end{verbatim}


% \section{Market Components}
% {\color{red} Market draft example for discussion ... }

% \subsection{Top-level Market}
% \begin{verbatim}
% "market”: {
%     "generator" : { ... },
%     "load" : { ... },
% }    
% \end{verbatim}
% \subsection{Generator}
% \begin{verbatim}
% "IDG0001" : {
%     "startup_costs": [300.0, 400.0, 500.0], 
%     "startup_delays": [4, 6, 8],
%     "bids": [{
%         "start_time": "2021-04-30T14:00",
%         "production_levels": [1.00, 1.10, 1.30, 1.35],
%         "production_costs": [1400.0, 1600.0, 2200.0, 2400.0],
%     },{
%         "start_time": "2021-04-30T15:00",
%         "production_levels": [1.00, 1.10, 1.30, 1.35],
%         "production_costs": [1460.0, 1670.0, 2250.0, 2430.0],
%     }]
% },
% "IDG0002” : { … },
% "IDG0003” : { … },
% …     
% \end{verbatim}
% \subsection{Load}
% \begin{verbatim}
% "IDL0001" : {
%     "bids": [{
%         "start_time": "2021-04-30T14:00",
%         "demand_levels": [0.10, 0.11, 0.15, 1.00],
%         "demand_price": [140.0, 150.0, 160.0, 200.0],
%     },{
%         "start_time": "2021-04-30T15:00",
%         "demand_levels": [0.10, 0.11, 0.15, 1.00],
%         "demand_price": [150.0, 170.0, 200.0, 220.0],
%     }]
% },
% "IDL0002” : { … },
% "IDL0003” : { … },
% …     
% \end{verbatim}
