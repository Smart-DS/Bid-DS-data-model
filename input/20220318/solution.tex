\label{sec:solution_format}
This section presents the latest data format in JSON for the 
communicating the solution data back to the ARPA-E competition team.
The section first presents the top level JSON object --- the
object for storing all the solution/output data 
%(attributes marked with `O'/`B' in Section~\ref{sec:naming})
for each network component.
JSON objects associated with each individual component will be presented afterwards.

\begin{todo}[]{}
Preliminary example
\end{todo}

\subsection{Solution Output Top Level}
\label{sec:output_data}

\begin{verbatim}
"solution_output”: {
   "bus": [ … ],
   "shunt": [ … ],
   "simple_dispatchable_device": [ … ],
   "multi-mode_dispatchable_device": [ … ],
   "sub_device": [ … ],
   "storage": [ … ],
   "ac_line”: [ … ],
   "two_winding_transformer”: [ … ],
   "dc_line”: [ … ],   
   "active_regional_reserve”: [ … ],
   "reactive_regional_reserve”: [ … ]
}    
\end{verbatim}

\subsection{Bus}
Attributes for buses are listed in Section~\ref{nom:bus}.
Below shows a schematic example on output parameters for buses.
\begin{verbatim}
{
  "uid" : "IDB0001",
  "vm" : [1.0, 0.99, 0.98],
  "va" : [0.0, 0.01, 0.02]
},
...
\end{verbatim}

\subsection{Shunt}
Attributes for shunts are listed in Section~\ref{nom:shunt}.
Below shows a schematic example on the output parameters for shunts.
\begin{verbatim}
{
    "uid" : "IDSH0001",
    "steps" : [1, 1, 0]
},
…    
\end{verbatim}

\subsection{Dispatchable Devices: Simple Producing \& Consuming Devices}
\begin{todo}[]{}
TBD
\end{todo}

\subsection{Dispatchable Devices: Multi-mode Producing \& Consuming Devices}
\begin{todo}[]{}
TBD
\end{todo}

\subsection{Subdevice Units}
\begin{todo}[]{}
TBD
\end{todo}

\subsection{Storage Device Units}
\begin{todo}[]{}
TBD
\end{todo}

% \begin{verbatim}
% {
%     "uid" : "IDG0001",
%     "status" : [true, true, true],
%     "select_config" : ["IDG0001-C1", "IDG0001-C2", "IDG0001-C2"],
%     "pg" : [0.0, 2.0, 7.0],
%     "qg" : [0.0, 1.0, 4.0],
%     "pg_regulation_down" : [0.0, 2.0, 5.0],
%     "pg_regulation_up" : [0.0, 0.0, 0.0],
%     "pg_spin" : [0.0, 0.0, 0.0],
%     "pg_cont" : [0.0, 3.0, 3.0],
%     "energy" " [0.0, 3.0, 3.0]        
% },
% …     
% \end{verbatim}

\subsection{AC Transmission Line}
Attributes for AC transmission lines are listed in Section~\ref{nom:line}.
Below shows a schematic example on the output parameters for lines.

\begin{verbatim}
{
    "uid" : "IDAC0001",
    "status" : [false, true, true],
    "pl_fr" : [0.0, 1.0, 1.0],
    "ql_fr" : [0.0, 0.5, 0.5],
    "pl_to" : [0.0, 0.95, 0.95],
    "ql_to" : [0.0, 0.49, 0.49]
},
…     
\end{verbatim}



\subsection{Two Winding Transformer}
Attributes for two winding transformers are listed in Section~\ref{nom:transformer}.
Below shows a schematic example on the output parameters for transformers.

\begin{verbatim}
{
    "uid" : "ID2T0001",
    "tm"    : [1.0, 1.05, 1.0],
    "ta"    : [0.0, 0.0, 0.0],
    "status" : [true, true, false],
    "pl_fr" : [1.0, 1.0, 0.0],
    "ql_fr" : [0.9, 1.0, 0.0],
    "pl_to" : [1.0, 1.0, 0.0],
    "ql_to" : [0.9, 1.0, 0.0]
},
…     
\end{verbatim}

\subsection{DC Line}
Attributes for DC lines are listed in Section~\ref{nom:dcline}.
Below shows a schematic example on the output parameters for DC lines.
\begin{verbatim}
{
   "uid" : "IDDC0001",
   "status" : [true, true, true],   
   "pdc_fr” : [0.5, 0.6, 0.7],
   "qdc_fr” : [0.0, 0.0, 0.1],
   "qdc_to” : [0.0, 0.0, -0.1]
},
…    
\end{verbatim}



\subsection{Regional Reserve}
Attributes for regional reserves are listed in Section~\ref{nom:reserves}.
Below shows a schematic example on the output parameters for active reserve zones.
\begin{verbatim}
{
  "uid" : "IDRES0001",
  "REG_UP" : [0.10, 0.10, 0.10],
  "REG_DOWN" : [0.05, 0.03, 0.05],
  "SYN" : [0.40, 0.40, 0.60],
  "NSYN" : [1.00, 1.00, 1.00],  
  "RAMPING_RESERVE_UP" : [2.00, 3.00, 4.00],  
  "RAMPING_RESERVE_DOWN" : [2.00, 3.00, 1.00]
},
...
\end{verbatim}

Below shows a schematic example on the output parameters for reactive reserve zones.
\begin{verbatim}
{
  "uid" : "IDRES0010",
  "REACT_UP" : [0.02, 0.02, 0.01],
  "REACT_DOWN" : [0.02, 0.02, 0.01]
},
...
\end{verbatim}

